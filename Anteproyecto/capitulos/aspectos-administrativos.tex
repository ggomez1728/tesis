%!TEX root = ../trabajo.tex
\capitulo{ASPECTOS ADMINISTRATIVOS}
\seccion{Plan de Trabajo}
A continuación se presenta una reseña de los recursos a utilizar y un cronograma de trabajo. En esta se detallan los recursos materiales y humanos requeridos en las fases y actividades previstas.
\seccion{Recursos}
A continuación se hace mención de los recursos utilizados para la realización del Trabajo de Grado.
	\subseccion{Humanos}
	Un Tutor académico, profesor MSc. y miembro de la línea de investigación Borrosidad y Sistemas Difusos del Decanato de Ciencias y Tecnología de la UCLA.
	\subseccion{Materiales}
	\begin{viñetas}
		\item Papelería.
		\item Ordenador.
		\item Manuales y guías para la elaboración de un trabajo de grado.
		\item Libros y artículos científicos en físico y digital, contentivos de los conocimientos y constructos necesarios para llevar a cabo la investigación.
		\item Cualquier otro elemento de apoyo pertinente a la investigación.
	\end{viñetas}
	\subseccion{Financieros}
	El financiamiento de la investigación estará a cargo del autor.
\seccion{Cronograma de Actividades: Diagrama de Gantt}
El cronograma de actividades se usa para determinar la fecha de terminación del trabajo, y debe permitir relacionar cada actividad con el correspondiente tiempo que implica su realización \cite[]{UNEXPOmanual2004}.
El \refcuadro{tabladiagrama} muestra el cronograma de actividades de la investigación. El mismo comprende las actividades requeridas para su desarrollo, indicadas en el procedimiento de la investigación en el capítulo 3. La fecha estimada de culminación es diciembre de 2016. 

\begin{cuadro}[titulo= Diagrama de Gantt, etiqueta = tabladiagrama]{|c|l|l|l|l|l|l|l|l|l|l|l|}

\hline
			  			  & \multicolumn{11}{c|}{\textbf{Meses 2016}}    \\ \hline
\textbf{Actividades}       & 2 & 3 & 4 & 5 & 6 & 7 & 8 & 9 & 10 & 11 & 12 \\ \hline
\textit{a}                 & X &   &   &   &   &   &   &   &    &    &    \\ \hline
\textit{b}                 & X &   &   &   &   &   &   &   &    &    &    \\ \hline
\textit{c}                 & X &   &   &   &   &   &   &   &    &    &    \\ \hline
\textit{d}                 & X & X &   &   &   &   &   &   &    &    &    \\ \hline
\textit{e}                 &   & X &   &   &   &   &   &   &    &    &    \\ \hline
\textit{f}                 &   & X & X &   &   &   &   &   &    &    &    \\ \hline
\textit{g}                 &   &   & X &   &   &   &   &   &    &    &    \\ \hline
\textit{h}                 &   &   & X & X &   &   &   &   &    &    &    \\ \hline
\textit{i}                 &   &   &   & X & X &   &   &   &    &    &    \\ \hline
\textit{j}                 &   &   &   &   & X & X &   &   &    &    &    \\ \hline
\textit{k}                 &   &   &   &   &   & X & X &   &    &    &    \\ \hline
\textit{l}                 &   &   &   &   &   &   & X & X &    &    &    \\ \hline
\textit{m}                 &   &   &   &   &   &   &   & X &  X &    &    \\ \hline
\textit{n}                 &   &   &   &   &   &   &   &   &  X & X  &    \\ \hline
\textit{o}                 &   &   &   &   &   &   &   &   &    & X  &  X  \\ \hline
\textit{p}                 &   &   &   &   &   &   &   &   &    &    &  X  \\ \hline

\end{cuadro}

