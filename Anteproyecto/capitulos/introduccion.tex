%!TEX root = ../trabajo.tex
\introduccion
Se estima que el crecimiento del mercado global de aplicaciones móviles de los últimos tres años se encuentre cerca del 210\%, representando un crecimiento de \$143 mil millones en el año 2016.  Esto se debe principalmente al uso cotidiano de los dispositivos móviles, lo que ha conducido a demandas más altas en cuanto a la calidad y estabilidad de productos ofertados. Además, este significativo crecimiento ha dado lugar a un aumento en la importancia de una mejor interacción entre el usuario y la tecnología. Sin embargo, las tendencias móviles suelen cambiar rápido debido a su estrecha relación con el hardware, el cual se encuentra en una continua renovación y ocasiona que los productos duren un corto período de tiempo. Adicionalmente la alta competitividad en el desarrollo de aplicaciones hace imperioso optimizar el uso de los recursos de una organización -tiempo, dinero y esfuerzo-, necesarios en el inicio del desarrollo de proyectos, por lo que la decisión de priorizar y evaluar proyectos es un punto álgido en toda organización dedicada al desarrollo de software. Razón por la que se propone en el presente trabajo de investigación el diseño de un algoritmo para sistemas de soporte de decisiones, que permita a los usuarios hacer elecciones de  criterios de selección adecuados para los proyectos de aplicaciones móviles, mediante el uso de valoraciones subjetivas e imprecisas, representadas mediante conjuntos borrosos.\\

Este proyecto de investigación está estructurado en cuatro capítulos, conforme al manual para la elaboración del trabajo conducente a grado académico de especialización, maestría y doctorado de la UCLA (2002). En el primero, el planteamiento del problema, se exponen los diferentes elementos que conforman el entorno de la situación planteada con respecto a los criterios requeridos, así como también a los modelos de toma de decisiones que podrán contribuir en el desarrollo del algoritmo para la clasificación y selección de proyectos de aplicaciones móviles. Igualmente, se describen los objetivos del mismo, la justificación,  importancia y alcance, elementos importantes que sirven de base y guía en toda investigación.\\

El Capítulo II, marco teórico, inicia con los antecedentes y las bases teóricas que apoyan el conocimiento del tema en estudio; se describen conceptos asociados con la lógica difusa, sistemas de información multicriterio, toma de decisiones multicriterio y multiatributos, además de las características y requerimientos principales de las Aplicaciones Móviles.\\

En el Capítulo III, marco metodológico, se describen actividades de las cuales depende la validez de la investigación como son: naturaleza del estudio, diseño de la investigación, así como los lineamientos de cómo se va a desarrollar la propuesta y la evaluación de la misma.\\

Finalmente en el Capítulo IV, aspectos administrativos, se presenta una reseña de los recursos a utilizar y el cronograma de trabajo para el desarrollo y conclusión de la presente investigación.\\

