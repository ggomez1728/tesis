%!TEX root = ../trabajo.tex
El significativo crecimiento del mercado de tecnologías móviles y la alta competitividad en el desarrollo de sus aplicaciones hace imperioso optimizar el uso de los recursos de una organización ---tiempo, dinero y esfuerzo---, necesarios en el inicio del desarrollo de proyectos, por lo que la decisión de priorizar y evaluar proyectos es un punto álgido en toda organización. El presente trabajo de investigación tiene como objetivo diseñar un algoritmo para sistemas de soporte de decisiones, que permita a los usuarios hacer elecciones con base en criterios subjetivos e imprecisos que serán representados mediante conjuntos difusos,  siendo más acordes a la naturaleza de los proyectos de aplicaciones móviles, al tratarse del desarrollo de nuevos productos. Esta investigación se hará bajo la modalidad de estudios de proyectos, fundamentada en investigación monográfica documental. Cabe resaltar que el presente trabajo estará orientado desde la perspectiva de la investigación tecnológica con un enfoque en la ciencia del diseño. En la propuesta se identificarán los criterios representativos y las tareas necesarias para la evaluación de proyectos de aplicaciones móviles, seguido del desarrollo de las etapas de análisis y conformación del algoritmo para la toma de decisiones en base a los criterios anteriormente definidos y finalmente se evaluará el algoritmo propuesto.